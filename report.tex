\documentclass[a4paper,11pt,twoside]{article}
%\documentclass[a4paper,11pt,twoside,se]{article}

\usepackage{UmUStudentReport}
\usepackage{verbatim}   % Multi-line comments using \begin{comment}
\usepackage{courier}    % Nicer fonts are used. (not necessary)
\usepackage{pslatex}    % Also nicer fonts. (not necessary)
\usepackage[pdftex]{graphicx}   % allows including pdf figures
\usepackage{listings}
\usepackage{pgf-umlcd}
\usepackage{blindtext}
\usepackage{rotating}
\usepackage{enumitem}
%\usepackage{lmodern}   % Optional fonts. (not necessary)
%\usepackage{tabularx}
%\usepackage{microtype} % Provides some typographic improvements over default settings
%\usepackage{placeins}  % For aligning images with \FloatBarrier
%\usepackage{booktabs}  % For nice-looking tables
%\usepackage{titlesec}  % More granular control of sections.

% DOCUMENT INFO
% =============
\department{Department of Computing Science}
\coursename{Operating Systems 7.5 p}
\coursecode{5DV171}
\title{Assignment 1}
\author{Lorenz Gerber ({\tt{dv15lgr@cs.umu.se}})}
\date{2017-09-23}
%\revisiondate{2016-01-18}
\instructor{Jan Erik Moström / Adam Dahlgren Lindström}


% DOCUMENT SETTINGS
% =================
\bibliographystyle{plain}
%\bibliographystyle{ieee}
\pagestyle{fancy}
\raggedbottom
\setcounter{secnumdepth}{2}
\setcounter{tocdepth}{2}
%\graphicspath{{images/}}   %Path for images

\usepackage{float}
\floatstyle{ruled}
\newfloat{listing}{thp}{lop}
\floatname{listing}{Listing}



% DEFINES
% =======
%\newcommand{\mycommand}{<latex code>}

% DOCUMENT
% ========
\begin{document}
\lstset{language=C}
\maketitle
\thispagestyle{empty}
%\newpage
%\tableofcontents
%\thispagestyle{empty}
\newpage

\clearpage
\pagenumbering{arabic}

\section{Introduction}
This assignment was about implementing a new system call in the Linux kernel.
The system call should gather information about number of processes, observed
file descriptors and pending signals for the current user. The information shall
be obtained in kernel space and then sent back to user space.

\section{Setup and Preparation}
A recent version of raspbian was installed on the SD card (`Raspian
Stretch Lite', Version September 2017, release 2017-09-07 with initial Kernel
version 4.9.) For performance and convenience, the kernel was cross-compiled on a
Linux machine (Ubuntu 16.04 LTS). Hence the tool-chain had to be set up for
cross compilation. The tool-chain was obtained from
\verb+git clone https://github.com/raspberrypi/tools+.
The Linux kernel source repository for Raspberry Pi Raspian was forked form
\verb+https://github.com/raspberrypi/linux+. For convenience, scripts for clean
up, configuration, build and deployment were setup
(\verb+https://github.com/lorenzgerber/kernel_build_scripts.git+).

\section{Implementation Syscall}
\subsection{Summary of Implementation}
All data needed to implement the required syscall is available in the
\verb+struct task_struct+ that is defined in the \verb+sched.h+ file. A step wise
implementation was followed where first a simple `Helloworld' syscall that
would print to the kernel message buffer was written. In a second iteration,
the data gathering which will be described in more detail below was implemented.
Finally, the data transfer between user and kernel space was coded. During all
steps, it was tried to use test and verification procedures to make sure that
the obtained data is correct.

\subsection{Used Data structures}
To get an idea of the involved data structures, initially chapter 3 of
\cite{bovetcesati2005} was consulted and compared with the data structures found
in the code. Using Eclipse as IDE proved to be of great use for getting pop-up
descriptions of macros and functions. Further, it greatly simplified navigating
the code by the possibility to link from implementations to definitions etc.

As mentioned earlier, the main data container to be used was the
\verb+struct task_struct+ that was obtained as a double linked list and could be
traversed to calculate the number of processes with the current uid.

Within the iteration loop, the number of watched file descriptors was also
obtained from the \verb+files+ struct that contained another struct \verb+fdtab+
with the field `open'.

The pending signals were extracted by bit shifting from the `personal' and `shared'
pending signal data structures (\verb+signal->shared_pending.signal.sig+,
\verb+pending.signal.sig+). A good description of the signal data structures was
found in \cite{love2010}.

\subsection{Files changed and added}
Below follows a list of files that were modified. The code is available on
\url{https://github.com/lorenzgerber/linux_kernel/tree/submission_1}.
\begin{enumerate}
  \item \textit{kernel/Makefile}, the new source file was included
  \item \textit{include/uapi/asm-generic/unistd.h}, a define in the arch generic
  unistd header
  \item \textit{arch/arm/include/uapi/unistd.h}, a define in the arch specific
  unistd header
  \item \textit{arch/arm/kernel/calls.S}, the call was added in arch specific
  the syscall table
  \item \textit{include/linux/syscalls.h}, a prototype was added in the syscalls
  header file
  \item \textit{init/Kconfig}, a reference to the configuration file was added
\end{enumerate}

The new files added were:
\begin{enumerate}
  \item \textit{kernel/Kconfig.processInfo}, the configuration file
  \item \textit{kernel/processInfo.c}, the source file
  \item \textit{include/linux/processInfo.h}, the headers
\end{enumerate}

\subsection{Syscall Return values and error codes}
All return values were verified with separate methods. Number of processes was
compared to using \verb+ps -u uid+. Starting and shutting down processes was
tested to check that the syscall's return value represented the changes.
Open file descriptors were verified using \verb+lsof -u uid | wc -l+. Also here
it was checked during run-time that opening/closing files was represented in the
result values.
Finally, for verifying the pending signals, two small c programs were written,
where one was only looping over a sleep(5) instruction for a number of times.

The other program was a wrapper that would start a program with blocked
\verb+SIGINT+ signal handler. Invoking the looper with the wrapper, then
pressing \verb+Ctrl-x+ resulted in pending signals, which was first checked from
the console using \verb+cat /proc/pid/status+. Then the corresponding procedure
was run while invoking the syscall to verify the correct values.


\subsection{Data transfer Kernel to User Space}
Only data transport from kernel to user space was needed here. This was
implemented using the \verb+put_user+ functions. The variables in user space
were provided to kernel space as arguments in the syscall. Hence
the syscall was implemented using the \verb'SYSCALL_DEFINE3' macro.

\subsection{User space code}
The user space stub does nothing else than setting up the variables to
receive data from kernel space, invoke the system call and finally print
the obtained information to stdout \ref{fig:userstbu}.

\begin{figure}
  \label{fig:sigpending}
  \centering
  \begin{verbatim}
#include <stdio.h>
#include <stdlib.h>
#include <linux/kernel.h>
#include <sys/syscall.h>
#include <unistd.h>

long processinfo_sys(int *procs, int* fds, int* sigpends){
	return syscall(398);
}

int main (int argc, char *argv[]){

	int procs = 0;
	int fds = 0;
	int sigpends = 0;

	long int a = processinfo_sys(&procs, &fds, &sigpends);

	printf("Processes: %d\n", procs);
	printf("Watched fds: %d\n", fds);
	printf("Pending Signals: %d\n", sigpends);
	printf("System call returned %ld\n", a);
	return 0;
}

  \end{verbatim}
  \caption{\textit{Code listing processInfo.c, the user space stub for
  invoking the system call. After setting up the variables where the data
  can be received, the program invokes the syscall by its number and finally
  prints out the obtained data to stdout.}}

\end{figure}

\addcontentsline{toc}{section}{\refname}
\bibliography{references}

\end{document}
