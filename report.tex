\documentclass[a4paper,11pt,twoside]{article}
%\documentclass[a4paper,11pt,twoside,se]{article}

\usepackage{UmUStudentReport}
\usepackage{verbatim}   % Multi-line comments using \begin{comment}
\usepackage{courier}    % Nicer fonts are used. (not necessary)
\usepackage{pslatex}    % Also nicer fonts. (not necessary)
\usepackage[pdftex]{graphicx}   % allows including pdf figures
\usepackage{listings}
\usepackage{pgf-umlcd}
\usepackage{blindtext}
\usepackage{rotating}
\usepackage{enumitem}
%\usepackage{lmodern}   % Optional fonts. (not necessary)
%\usepackage{tabularx}
%\usepackage{microtype} % Provides some typographic improvements over default settings
%\usepackage{placeins}  % For aligning images with \FloatBarrier
%\usepackage{booktabs}  % For nice-looking tables
%\usepackage{titlesec}  % More granular control of sections.

% DOCUMENT INFO
% =============
\department{Department of Computing Science}
\coursename{Operating Systems 7.5 p}
\coursecode{5DV171}
\title{Assignment 1}
\author{Lorenz Gerber ({\tt{dv15lgr@cs.umu.se}} {\tt{lozger03@student.umu.se}})}
\date{2017-09-26}
%\revisiondate{2016-01-18}
\instructor{Jan Erik Moström / Adam Dahlgren Lindström}


% DOCUMENT SETTINGS
% =================
\bibliographystyle{plain}
%\bibliographystyle{ieee}
\pagestyle{fancy}
\raggedbottom
\setcounter{secnumdepth}{2}
\setcounter{tocdepth}{2}
%\graphicspath{{images/}}   %Path for images

\usepackage{float}
\floatstyle{ruled}
\newfloat{listing}{thp}{lop}
\floatname{listing}{Listing}



% DEFINES
% =======
%\newcommand{\mycommand}{<latex code>}

% DOCUMENT
% ========
\begin{document}
\lstset{language=C}
\maketitle
\thispagestyle{empty}
%\newpage
%\tableofcontents
%\thispagestyle{empty}
\newpage

\clearpage
\pagenumbering{arabic}

\section{Introduction}

\section{Step by Step Description}
\subsection{Preparing Raspberry Pi}
First a recent version of raspbian was installed on the SD card. Here we used `Raspian
Stretch Lite', Version September 2017, release 2017-09-07 with initial Kernel version
4.9.

\subsection{Setting up Toolchain}
Here it was decided to cross-compile the kernel from a Ubunut Linux machine. Hence
the toolchain had to be set up for cross compiling. The toolchain was obtained
from \verb+git clone https://github.com/raspberrypi/tools+. In the instructions
(\verb+https://www.raspberrypi.org/documentation/linux/kernel/building.md+), it
was advised to copy the toolchain to a common location and add the directory
\verb+/tools/arm-bcm2708/gcc-linaro-arm-linux-gnueabihf-raspbian-x64/bin+ to
the \verb+.bashrc+ path.
\subsection{Getting the source}
The Linux kernal sources for Raspberry Pi Raspian were obtained from
\verb+git clone --depth=1 https://github.com/raspberrypi/linux+.

\subsection{Test Building the Kernel}
\begin{verbatim}
cd linux
KERNEL=kernel7
make ARCH=arm CROSS_COMPILE=arm-linux-gnueabihf- bcm2709_defconfig
make -j 4 -ARCH=arm CROSS_COMPILE=arm-linux-gnueabihf- zImage modules dtbs
\end{verbatim}

Instead of setting the \verb+.bashrc+ path, it is also possible to indicate
the full path for \verb+arm-linux-gnueabihf+. In this case, this was
\verb+/usr/local/tools/arm-bcm2708/gcc-linaro-arm-linux-gnueabihf-raspbian-x64/bin/+.
The \verb+-j 4+ flag chooses the number of processers (times 1.5) to use.

Now the SD card of the Raspberry is plugged into the Linux machine. Within
the Linux Kernel repo base directory, two mountpoints for the SD card partitions
are created (\verb+mkdir -p mnt/fat32+ and \verb+mkdir -p mnt/ext4+). The
SD card is mounted (\verb+sudo mount /dev/sdb1 mnt/fat32+ and
\verb+sudo mount /dev/sdb2/ mnt/ext4+. Then standing at the kernel repo base
directory, the modules are installed by:
\verb+sudo make ARCH=arm CROSS_COMPILE=arm-linux-gnueabihf- INSTALL_MOD_PATH=mnt/ext4 modules_install+.
Again, the full path to \verb+arm-linux-gnueabihf-+ was needed.
Then the kernel and device tree blobs were copied onto the SD card, also backing
up the old kernel:
\begin{verbatim}
sudo cp mnt/fat32/$KERNEL.img mnt/fat32/$KERNEL-backup.img
sudo cp arch/arm/boot/zImage mnt/fat32/$KERNEL.img
sudo cp arch/arm/boot/dts/*.dtb mnt/fat32/
sudo cp arch/arm/boot/dts/overlays/*.dtb* mnt/fat32/overlays/
sudo cp arch/arm/boot/dts/overlays/README mnt/fat32/overlays/
\end{verbatim}

Then the SD card can be umounted and booted in the Raspberry Pi.






\addcontentsline{toc}{section}{\refname}
\bibliography{references}

\end{document}
